% Universal Workflow System - PROMISE 2026 Version
% Target Venue: PROMISE 2026 (Predictive Models and Data Analytics in SE)
% Format: ACM sigconf (FSE Companion Proceedings)

\documentclass[acmsmall,screen,review,anonymous]{acmart}

% ============================================================================
% PACKAGES
% ============================================================================
\usepackage{booktabs}       % Professional tables
\usepackage{subcaption}     % Subfigures
\usepackage{listings}       % Code listings
\usepackage{xcolor}         % Colors
\usepackage{tikz}           % Diagrams
\usepackage{pgfplots}       % Plots
\usepackage{algorithm}      % Algorithms
\usepackage{algorithmic}    % Algorithm formatting
\usepackage{balance}        % Balance columns on last page

\pgfplotsset{compat=1.18}

% ============================================================================
% LISTINGS CONFIGURATION
% ============================================================================
\lstset{
    basicstyle=\ttfamily\small,
    breaklines=true,
    frame=single,
    numbers=left,
    numberstyle=\tiny\color{gray},
    keywordstyle=\color{blue},
    commentstyle=\color{green!50!black},
    stringstyle=\color{red},
    showstringspaces=false,
    tabsize=2
}

% Define bash language for listings
\lstdefinelanguage{bash}{
    keywords={if, then, else, elif, fi, for, while, do, done, case, esac, function, return, local, export},
    morecomment=[l]{\#},
    morestring=[b]",
    morestring=[b]'
}

% ============================================================================
% METADATA
% ============================================================================
\title{Predicting Context Recovery Success in AI-Assisted Development}

% Authors (anonymized for review)
\author{Anonymous Author(s)}
\affiliation{
    \institution{Anonymous Institution}
    \city{City}
    \country{Country}
}
\email{anonymous@example.com}

% ============================================================================
% ABSTRACT
% ============================================================================
\begin{abstract}
AI-assisted software development has transformed how developers work, yet context loss during extended projects remains a critical challenge. When sessions span weeks or months, developers spend 15--25 minutes recovering context after each interruption. While workflow management systems can reduce this overhead, a fundamental question remains: Can we predict whether context recovery will succeed, and how long it will take?

We present the first predictive models for workflow context recovery in AI-assisted development. Using the Universal Workflow System (UWS) as our experimental platform, we generate a dataset of 3,000 recovery scenarios spanning diverse conditions: varying checkpoint counts (1--200), state complexities (minimal to complex), corruption levels (0--90\%), and interruption types (clean, crash, timeout). We train and evaluate machine learning models to predict recovery time, success probability, and state completeness.

Our results demonstrate that workflow recovery is predictable: Gradient Boosting achieves MAE of 1.1ms for recovery time prediction ($R^2 = 0.756$) and AUC-ROC of 0.912 for recovery success classification. Feature importance analysis reveals that corruption level ($r = -0.475$), checkpoint characteristics, and handoff document size are the dominant factors. We release a public benchmark dataset enabling future research on development workflow analytics.

\end{abstract}

% ============================================================================
% CCS CONCEPTS AND KEYWORDS
% ============================================================================
\begin{CCSXML}
<ccs2012>
   <concept>
       <concept_id>10011007.10011006.10011008.10011009.10011012</concept_id>
       <concept_desc>Software and its engineering~Software development techniques</concept_desc>
       <concept_significance>500</concept_significance>
   </concept>
   <concept>
       <concept_id>10010147.10010257.10010293.10010294</concept_id>
       <concept_desc>Computing methodologies~Supervised learning by regression</concept_desc>
       <concept_significance>500</concept_significance>
   </concept>
</ccs2012>
\end{CCSXML}

\ccsdesc[500]{Software and its engineering~Software development techniques}
\ccsdesc[500]{Computing methodologies~Supervised learning by regression}

\keywords{predictive models, workflow recovery, context management, machine learning, software analytics}

% ============================================================================
% DOCUMENT
% ============================================================================
\begin{document}

\maketitle

% Include PROMISE-specific sections
% ============================================================================
% INTRODUCTION - PROMISE 2026 VERSION
% Focused on Predictive Models for Context Recovery
% ============================================================================
\section{Introduction}
\label{sec:introduction}

AI-assisted software development has fundamentally transformed how developers approach complex tasks. Large language models (LLMs) serve as intelligent coding assistants for code generation, debugging, and project planning. However, this paradigm introduces a critical operational challenge: \textit{context persistence across extended development sessions}. When developers return to a project after interruptions---meetings, context window limits, or system restarts---workflow management systems must restore saved state from checkpoints, a process that can fail due to file corruption, incomplete writes, or schema changes.

Prior research establishes that developers require 15--25 minutes to cognitively re-engage after interruptions~\cite{mark2008cost, parnin2011programmer}. While workflow tools aim to reduce this burden by preserving technical state, \textbf{we do not claim to measure or predict developer cognitive recovery}---that remains an open research question requiring user studies. Instead, we address a prerequisite question: \textbf{Can we predict whether \textit{system-level} state recovery will succeed, and how long it will take?} Such predictions enable adaptive checkpointing, proactive failure warnings, and informed tool reliability assessments.

This paper presents the first predictive models for workflow context recovery in AI-assisted development, along with a \textbf{synthetic benchmark for reproducible research} on workflow resilience. Using the Universal Workflow System (UWS) as our experimental platform, we generate a comprehensive dataset of 3,000 recovery scenarios spanning diverse conditions: varying checkpoint counts (1--200), state complexities (minimal to complex), corruption levels (0--90\%), and interruption types (clean, abrupt, crash, timeout).

Unlike observational datasets (e.g., KaVE~\cite{amann2018feedbag}, DevGPT~\cite{xiao2024devgpt}) that capture developer behavior without controlled interventions, our synthetic benchmark enables \textit{causal analysis} of how specific factors (corruption level, state complexity) affect recovery outcomes. We train and evaluate multiple machine learning models to predict:

\begin{enumerate}
    \item \textbf{System Recovery Time} (regression): How long will file parsing and state restoration take?
    \item \textbf{Recovery Success} (classification): Will state restoration succeed under given corruption conditions?
    \item \textbf{State Completeness} (regression): What percentage of checkpoint data will be successfully parsed?
\end{enumerate}

Our key findings demonstrate that \textit{system-level} workflow recovery is predictable:

\begin{itemize}
    \item \textbf{Recovery time prediction} achieves MAE of 1.1ms using Gradient Boosting ($R^2 = 0.756$), enabling accurate estimation of file I/O and parsing overhead.
    \item \textbf{Recovery success prediction} achieves AUC-ROC of 0.912 and F1 of 0.911, enabling reliable failure anticipation before developers encounter errors.
    \item \textbf{Feature importance analysis} reveals that corruption level, checkpoint count, and handoff document size are the dominant factors affecting system recovery outcomes.
\end{itemize}

\paragraph{Contributions} This paper makes the following contributions:

\begin{enumerate}
    \item \textbf{Synthetic benchmark for reproducible research} on system-level workflow recovery: 3,000 annotated scenarios with controlled corruption levels, enabling causal analysis impossible with observational data. Unlike existing datasets (KaVE, DevGPT), our benchmark provides ground-truth \textit{system} recovery outcomes under systematic state degradation.

    \item \textbf{First predictive models} for system-level workflow state recovery, achieving MAE of 1.1ms for file parsing time and AUC-ROC of 0.912 for recovery success prediction. We explicitly scope this to technical state restoration, not developer cognitive recovery.

    \item \textbf{Feature importance analysis} identifying key factors affecting system recovery outcomes: corruption level ($r = -0.475$), checkpoint characteristics ($r = 0.318$), and handoff document size ($r = 0.531$).

    \item \textbf{Open-source implementation} of the Universal Workflow System with comprehensive test suite (356 tests: 93\% core, 76\% experimental), benchmark scripts, and replication package.
\end{enumerate}

\paragraph{PROMISE Relevance} This work contributes directly to the PROMISE community's mission of advancing predictive models and data analytics in software engineering. We build \textit{predictive models} for system-level recovery outcomes---enabling proactive failure warnings and adaptive checkpointing rather than reactive debugging. Our synthetic benchmark enables controlled experiments on workflow state restoration that are impossible with observational datasets. This positions our work squarely within PROMISE's scope: machine learning models trained on SE-specific data to predict software tool reliability. Note: We predict \textit{system} recovery (file I/O, parsing), not \textit{developer} cognitive recovery---the relationship between the two requires user studies we identify as future work.

\paragraph{Paper Organization} Section~\ref{sec:background} provides background on workflow systems and context management. Section~\ref{sec:approach} describes our predictive modeling approach. Section~\ref{sec:evaluation} presents evaluation methodology and results. Section~\ref{sec:related} discusses related work, and Section~\ref{sec:conclusion} concludes with implications for practice and future directions.

% ============================================================================
% BACKGROUND
% ============================================================================
\section{Background}
\label{sec:background}

This section provides context for understanding UWS's design and contributions. We define the context persistence problem, describe the landscape of existing solutions, and introduce a running example.

\subsection{The Context Persistence Problem}

When developers work with AI assistants on complex projects, they establish a shared understanding of the project's state, goals, and history. This \textit{context} encompasses:

\begin{itemize}
    \item \textbf{Project state}: Current phase, completed tasks, pending work
    \item \textbf{Design decisions}: Architectural choices and their rationale
    \item \textbf{Domain knowledge}: Project-specific concepts and terminology
    \item \textbf{Conversation history}: Previous interactions and insights
\end{itemize}

Context is fragile. LLM context windows have finite capacity (typically 8K--128K tokens). Session timeouts, system restarts, and natural work interruptions regularly reset this accumulated knowledge. Research shows developers require 15--25 minutes to regain focus after interruptions~\cite{mark2008cost, parnin2011programmer}, and AI-assisted workflows face similar challenges.

\subsection{Running Example: ML Research Workflow}

Consider Alice, a researcher developing a neural network for image classification. Her workflow spans several weeks:

\begin{enumerate}
    \item \textbf{Week 1}: Literature review, hypothesis formulation
    \item \textbf{Week 2}: Experimental design, data preparation
    \item \textbf{Week 3}: Model implementation, initial training
    \item \textbf{Week 4}: Hyperparameter tuning, validation
    \item \textbf{Week 5}: Results analysis, paper writing
\end{enumerate}

Each phase involves different skills and context. Without persistence, Alice would repeatedly explain her dataset, model architecture, and experimental results to her AI assistant. With UWS, each session resumes from a checkpoint containing all relevant state.

\subsection{State Management Approaches}

Existing systems manage state in various ways:

\paragraph{Database-Backed State} Systems like Airflow~\cite{hazelwood2018airflow} and Temporal~\cite{temporal2020} store state in relational databases (PostgreSQL, MySQL). This enables scalable, distributed execution but requires infrastructure setup and lacks version control semantics.

\paragraph{In-Memory State} Many agent frameworks (LangChain~\cite{chase2022langchain}, AutoGen~\cite{autogen2023}) maintain state in memory with optional persistence to external stores. Recovery typically involves replaying conversation history, which can be slow and lose nuanced context.

\paragraph{File-Based State} Some tools use JSON or YAML files for state persistence. While simple, these approaches rarely integrate with version control systems and lack standardized recovery mechanisms.

\paragraph{Git-Native State (UWS's Novel Approach)} UWS introduces a design innovation: storing \emph{all} workflow state in version-controlled YAML files within the project's git repository. Unlike prior approaches, UWS combines three elements that, to our knowledge, have not been integrated before:

\begin{enumerate}
    \item \textbf{Dual-layer persistence}: Structured state (YAML) paired with human-readable handoff (Markdown), enabling both machine parsing and human comprehension.
    \item \textbf{Fallback recovery}: When primary state files are corrupted, UWS automatically attempts recovery from the human-readable handoff document, providing graceful degradation.
    \item \textbf{Zero-infrastructure design}: No external database, message queue, or container runtime required---only git and standard Unix utilities (bash, grep, sed).
\end{enumerate}

This design enables:
\begin{itemize}
    \item \textbf{Natural versioning}: Every checkpoint is a git commit
    \item \textbf{Diffing}: State changes visible via \texttt{git diff}
    \item \textbf{Branching}: Experimental workflows via git branches
    \item \textbf{Portability}: Works on any system with git and bash
\end{itemize}

The key insight is that by treating workflow state as \emph{source code}---versioned, diffable, and human-readable---we inherit decades of version control best practices for free.

\subsection{Workflow Phases}

UWS organizes work into five standard phases, adaptable to any project type:

\begin{enumerate}
    \item \textbf{Planning}: Requirements gathering, scope definition, design
    \item \textbf{Implementation}: Code development, model building
    \item \textbf{Validation}: Testing, experiments, verification
    \item \textbf{Delivery}: Deployment, documentation, release
    \item \textbf{Maintenance}: Monitoring, updates, support
\end{enumerate}

Each phase has associated deliverables and completion criteria. Agents specialize in particular phases, and skills are organized by relevance to each phase.

\subsection{Multi-Agent Workflows}

Modern AI development benefits from specialized agents~\cite{wu2023autogen, hong2024metagpt}. UWS defines seven agent roles:

\begin{itemize}
    \item \textbf{Researcher}: Literature review, hypothesis formation
    \item \textbf{Architect}: System design, API specification
    \item \textbf{Implementer}: Code development, testing
    \item \textbf{Experimenter}: Running experiments, benchmarks
    \item \textbf{Optimizer}: Performance tuning, profiling
    \item \textbf{Deployer}: CI/CD, containerization, monitoring
    \item \textbf{Documenter}: Documentation, paper writing
\end{itemize}

Transitions between agents follow defined patterns with explicit handoff protocols, ensuring context preservation across role changes.

% ============================================================================
% APPROACH
% ============================================================================
\section{Approach}
\label{sec:approach}

This section describes UWS's architecture, design decisions, and key components. We emphasize the technical challenges solved and the rationale behind design choices.

\subsection{Architecture Overview}

Figure~\ref{fig:architecture} shows UWS's high-level architecture. The system consists of four main components:

\begin{enumerate}
    \item \textbf{State Management Layer}: Git-native persistence via YAML files
    \item \textbf{Agent System}: Seven specialized agents with handoff protocols
    \item \textbf{Skill Library}: Modular capabilities loaded dynamically
    \item \textbf{Workflow Engine}: Phase-based execution with checkpoint integration
\end{enumerate}

% Architecture diagram placeholder
\begin{figure}[t]
    \centering
    \fbox{\parbox{0.9\columnwidth}{
        \centering
        \textbf{Architecture Diagram}\\[1em]
        \small
        [State Layer] $\rightarrow$ [Agent System] $\rightarrow$ [Skill Library]\\
        $\downarrow$\\
        [Git Repository] $\leftarrow$ [Checkpoint System]
    }}
    \caption{UWS Architecture: Git-native state persistence enables checkpoint-based recovery across sessions.}
    \label{fig:architecture}
\end{figure}

\subsection{Git-Native State Persistence}

UWS's core innovation is storing all workflow state in version-controlled files:

\paragraph{State Files} The \texttt{.workflow/} directory contains:
\begin{itemize}
    \item \texttt{state.yaml}: Current phase, checkpoint ID, metadata
    \item \texttt{config.yaml}: Project configuration, enabled features
    \item \texttt{checkpoints.log}: Timestamped checkpoint history
    \item \texttt{handoff.md}: Human-readable context for session continuity
\end{itemize}

\paragraph{Checkpoint Mechanism} Creating a checkpoint involves:
\begin{enumerate}
    \item Updating \texttt{state.yaml} with current context
    \item Appending to \texttt{checkpoints.log} with timestamp and description
    \item Optionally committing changes via git
\end{enumerate}

Listing~\ref{lst:checkpoint} shows the checkpoint creation process:

\begin{lstlisting}[language=bash,caption={Checkpoint creation},label={lst:checkpoint}]
# Create checkpoint with descriptive message
./scripts/checkpoint.sh "Completed model training"

# Checkpoint log format:
# 2024-01-15T14:30:00Z | CP_2_5 | Completed model training
\end{lstlisting}

\paragraph{Recovery Process} Context recovery reads checkpoint files and restores state:
\begin{enumerate}
    \item Load \texttt{state.yaml} to determine current phase
    \item Read \texttt{checkpoints.log} to show recent history
    \item Parse \texttt{handoff.md} for critical context
    \item Display suggested next actions
\end{enumerate}

The \texttt{recover\_context.sh} script completes this in under 5 minutes, compared to 15+ minutes for manual context rebuilding.

\subsection{Technical Challenges}

Implementing git-native state persistence required solving several challenges:

\paragraph{State Serialization} Agent context must be serializable to YAML. We limit state to:
\begin{itemize}
    \item Primitive types (strings, numbers, booleans)
    \item Nested structures (maps, lists)
    \item References to files (paths, not contents)
\end{itemize}

Large data (models, datasets) remain external; only references are checkpointed.

\paragraph{Atomic Operations} Checkpoint creation must be atomic to prevent corruption. We use:
\begin{itemize}
    \item Write to temporary file, then rename (atomic on POSIX)
    \item Validate YAML syntax before committing
    \item Backup previous state before modification
\end{itemize}

\paragraph{Conflict Resolution} Collaborative workflows may create git conflicts in state files. UWS:
\begin{itemize}
    \item Uses YAML's line-based format for easier merging
    \item Provides conflict detection warnings
    \item Suggests resolution strategies
\end{itemize}

\subsection{Multi-Agent System}

UWS's agent system implements role-based collaboration:

\paragraph{Agent Registry} The \texttt{.workflow/agents/registry.yaml} defines available agents with:
\begin{itemize}
    \item Capabilities and specializations
    \item Default skills
    \item Transition rules
\end{itemize}

\paragraph{Activation} Agents are activated via:
\begin{lstlisting}[language=bash]
./scripts/activate_agent.sh researcher
\end{lstlisting}

This updates \texttt{.workflow/agents/active.yaml} with the current agent, loads relevant skills, and creates the agent's workspace directory.

\paragraph{Handoff Protocol} Transitioning between agents follows a structured protocol:
\begin{enumerate}
    \item Current agent creates handoff notes
    \item Checkpoint is created with transition context
    \item New agent is activated
    \item New agent loads predecessor's notes
\end{enumerate}

\subsection{Skill Library}

Skills are modular capabilities that agents can use:

\paragraph{Skill Catalog} The \texttt{.workflow/skills/catalog.yaml} organizes skills by category:
\begin{itemize}
    \item Research: literature\_review, experimental\_design
    \item Development: code\_development, testing
    \item ML/AI: model\_training, hyperparameter\_tuning
    \item Deployment: containerization, ci\_cd
\end{itemize}

\paragraph{Skill Chains} Complex workflows use predefined chains:
\begin{itemize}
    \item \texttt{full\_research\_pipeline}: literature $\rightarrow$ design $\rightarrow$ validation
    \item \texttt{ml\_optimization}: profiling $\rightarrow$ quantization $\rightarrow$ benchmarking
\end{itemize}

\paragraph{Dynamic Loading} Skills are enabled via:
\begin{lstlisting}[language=bash]
./scripts/enable_skill.sh quantization pruning
\end{lstlisting}

This updates \texttt{.workflow/skills/enabled.yaml} and makes capabilities available to the active agent.

\subsection{Phase-Based Execution}

UWS structures work into phases:

\paragraph{Phase Transitions} Moving between phases:
\begin{enumerate}
    \item Verify phase completion criteria
    \item Create transition checkpoint
    \item Update \texttt{state.yaml} with new phase
    \item Activate appropriate agent for new phase
\end{enumerate}

\paragraph{Phase Workspaces} Each phase has a dedicated directory:
\begin{itemize}
    \item \texttt{phases/phase\_1\_planning/}: Requirements, scope docs
    \item \texttt{phases/phase\_2\_implementation/}: Source code
    \item \texttt{phases/phase\_3\_validation/}: Test results, metrics
\end{itemize}

\subsection{Implementation Details}

UWS is implemented in Bash for maximum portability:

\paragraph{Dependencies} Required: Bash 4+, git. Optional: yq (YAML processing).

\paragraph{Utility Libraries} Core functionality is provided by:
\begin{itemize}
    \item \texttt{yaml\_utils.sh}: YAML parsing with yq fallback
    \item \texttt{validation\_utils.sh}: Input validation and sanitization
\end{itemize}

\paragraph{Error Handling} All scripts use \texttt{set -euo pipefail} for strict error handling, with graceful degradation for optional features.

\subsection{Usage Example}

A typical UWS session:

\begin{lstlisting}[language=bash]
# Initialize workflow
./scripts/init_workflow.sh

# Activate researcher agent
./scripts/activate_agent.sh researcher

# Work on literature review...
# Create checkpoint
./scripts/checkpoint.sh "Literature review complete"

# Transition to implementer
./scripts/activate_agent.sh implementer

# After a break, recover context
./scripts/recover_context.sh
\end{lstlisting}

Recovery displays current state, recent checkpoints, active agent, and suggested next actions---enabling quick resumption of work.

% ============================================================================
% EVALUATION - PROMISE 2026 VERSION
% Focused on Predictive Model Evaluation
% ============================================================================
\section{Evaluation}
\label{sec:evaluation}

We evaluate our predictive models for workflow context recovery through comprehensive experiments addressing three research questions.

\subsection{Research Questions}

\begin{itemize}
    \item \textbf{RQ1 (Recovery Time Prediction)}: How accurately can we predict context recovery time from workflow characteristics?
    \item \textbf{RQ2 (Recovery Success Prediction)}: How accurately can we predict whether recovery will succeed under given conditions?
    \item \textbf{RQ3 (Feature Importance)}: Which workflow characteristics most strongly influence recovery outcomes?
\end{itemize}

\subsection{Dataset Construction}

\paragraph{Scenario Generation} We generated 1,000 unique recovery scenarios by systematically varying workflow parameters:

\begin{itemize}
    \item \textbf{Checkpoint count}: 1, 5, 10, 25, 50, 100, 200
    \item \textbf{State complexity}: minimal, low, medium, high, complex
    \item \textbf{Project type}: ML pipeline, web development, research, DevOps, data engineering, LLM application, mixed
    \item \textbf{Corruption level}: 0\%, 5\%, 10\%, 25\%, 50\%, 75\%, 90\%
    \item \textbf{Interruption type}: clean, abrupt, crash, timeout
    \item \textbf{Handoff size}: small (500 chars), medium (2,000), large (8,000), very large (25,000)
\end{itemize}

\paragraph{Ground Truth Collection} For each scenario, we executed three independent trials, measuring:
\begin{itemize}
    \item Recovery time (milliseconds)
    \item Recovery success (binary)
    \item State completeness (percentage)
\end{itemize}

The final dataset contains 3,000 entries (1,000 scenarios $\times$ 3 trials) with 18 features per entry.

\paragraph{Dataset Statistics} Table~\ref{tab:dataset-stats} summarizes the dataset characteristics:

\begin{table}[h]
    \centering
    \caption{Predictive Dataset Statistics}
    \label{tab:dataset-stats}
    \begin{tabular}{lr}
        \toprule
        \textbf{Metric} & \textbf{Value} \\
        \midrule
        Total entries & 3,000 \\
        Unique scenarios & 1,000 \\
        Features & 18 \\
        Recovery time (mean) & 29.4ms \\
        Recovery time (std dev) & 3.5ms \\
        Success rate & 85.3\% \\
        \bottomrule
    \end{tabular}
\end{table}

\subsection{Experimental Setup}

\paragraph{Models} We evaluated four model families using scikit-learn~\cite{pedregosa2011scikit}:
\begin{itemize}
    \item \textbf{Linear models}: Linear Regression, Ridge Regression, Logistic Regression
    \item \textbf{Ensemble models}: Random Forest~\cite{breiman2001random}, Gradient Boosting~\cite{friedman2001greedy}
\end{itemize}

\paragraph{Validation} We used 5-fold cross-validation on 80\% training data, with 20\% held out for final testing. All experiments used random seed 42 for reproducibility.

\paragraph{Metrics} For regression: MAE, RMSE, $R^2$. For classification: Accuracy, F1-score, AUC-ROC. All metrics include 95\% confidence intervals.

\subsection{RQ1: Recovery Time Prediction}

Table~\ref{tab:regression-results} presents recovery time prediction results:

% Auto-generated table for PROMISE 2026
\begin{table}[t]
    \centering
    \caption{Recovery Time Prediction Results (5-fold CV)}
    \label{tab:regression-results}
    \begin{tabular}{lrrr}
        \toprule
        \textbf{Model} & \textbf{MAE (ms)} & \textbf{RMSE (ms)} & \textbf{$R^2$} \\
        \midrule
        Linear Regression & 2.2 [2.1, 2.3] & 3.1 & 0.152 \\
        Ridge Regression & 2.2 [2.1, 2.3] & 3.1 & 0.152 \\
        Random Forest & 1.2 [1.1, 1.3] & 1.9 & 0.718 \\
        Gradient Boosting & 1.1 [1.0, 1.2] & 1.7 & 0.756 \\
        \bottomrule
    \end{tabular}
\end{table}


\paragraph{Findings} Gradient Boosting achieves the best performance with MAE of 1.1ms (95\% CI: [1.0, 1.2]) and $R^2 = 0.756$. This means:
\begin{itemize}
    \item Predictions are within 1.1ms of actual recovery time on average
    \item The model explains 75.6\% of variance in recovery time
    \item Linear models perform poorly ($R^2 = 0.152$), suggesting non-linear relationships
\end{itemize}

\paragraph{Practical Significance} With mean recovery time of 29.4ms, MAE of 1.1ms represents 3.7\% error---sufficient for practical applications like adaptive checkpointing and resource planning.

\subsection{RQ2: Recovery Success Prediction}

Table~\ref{tab:classification-results} presents recovery success prediction results:

% Auto-generated table for PROMISE 2026
\begin{table}[t]
    \centering
    \caption{Recovery Success Prediction Results (5-fold CV)}
    \label{tab:classification-results}
    \begin{tabular}{lrrr}
        \toprule
        \textbf{Model} & \textbf{Accuracy} & \textbf{F1 Score} & \textbf{AUC-ROC} \\
        \midrule
        Logistic Regression & 0.856 & 0.917 & 0.904 \\
        Random Forest & 0.845 & 0.909 & 0.907 \\
        Gradient Boosting & 0.851 & 0.911 & 0.912 \\
        \bottomrule
    \end{tabular}
\end{table}


\paragraph{Findings} All models achieve strong predictive performance:
\begin{itemize}
    \item Gradient Boosting: AUC-ROC = 0.912, F1 = 0.911
    \item Logistic Regression: AUC-ROC = 0.904, F1 = 0.917
    \item Random Forest: AUC-ROC = 0.907, F1 = 0.909
\end{itemize}

\paragraph{Confusion Matrix Analysis} On the test set (600 samples), Gradient Boosting achieved:
\begin{itemize}
    \item True Negatives: 51 (correctly predicted failures)
    \item False Positives: 37 (predicted success, actual failure)
    \item False Negatives: 49 (predicted failure, actual success)
    \item True Positives: 463 (correctly predicted successes)
\end{itemize}

The 5.8\% false positive rate indicates the model may occasionally overestimate recovery success under adverse conditions.

\subsection{RQ3: Feature Importance Analysis}

We analyzed feature importance using two methods: (1) Spearman correlation with target variables, and (2) model-based feature importance from tree ensembles.

\paragraph{Recovery Time Features} The top predictors of recovery time are:
\begin{enumerate}
    \item \textbf{Handoff document size} ($r = 0.531$, $p < 0.001$): Larger handoff documents increase recovery time
    \item \textbf{Checkpoint count} ($r = 0.318$, $p < 0.001$): More checkpoints require more parsing
    \item \textbf{Checkpoint log size} ($r = 0.318$, $p < 0.001$): Directly correlated with checkpoint count
\end{enumerate}

\paragraph{Recovery Success Features} The top predictors of recovery success are:
\begin{enumerate}
    \item \textbf{Corruption level} ($r = -0.475$, $p < 0.001$): Higher corruption strongly reduces success probability
    \item \textbf{Interruption type}: Crash and timeout interruptions reduce success rates
    \item \textbf{Phase progress}: Earlier phases show slightly higher recovery success
\end{enumerate}

\paragraph{Model-Based Importance} Random Forest feature importance confirms these findings:
\begin{itemize}
    \item For recovery time: checkpoint\_count (23\%), handoff\_chars (18\%), checkpoint\_log\_size (17\%)
    \item For recovery success: corruption\_level (31\%), interruption\_type (22\%), phase\_progress (15\%)
\end{itemize}

\subsection{State Completeness Prediction}

We also evaluated models for predicting state completeness percentage:

\begin{table}[h]
    \centering
    \caption{State Completeness Prediction Results}
    \label{tab:completeness-results}
    \begin{tabular}{lrrr}
        \toprule
        \textbf{Model} & \textbf{MAE (\%)} & \textbf{RMSE (\%)} & \textbf{$R^2$} \\
        \midrule
        Random Forest & 8.51 & 13.53 & 0.680 \\
        Gradient Boosting & 8.79 & 11.45 & 0.770 \\
        \bottomrule
    \end{tabular}
\end{table}

Gradient Boosting achieves MAE of 8.79\% for state completeness prediction ($R^2 = 0.770$), enabling estimation of how much context will be recoverable under given conditions.

\subsection{Threats to Validity}

\paragraph{Internal Validity} Scenario generation uses controlled parameter variation, which may not capture all real-world correlations. We mitigate this through systematic coverage of the parameter space combined with random sampling.

\paragraph{External Validity} Our dataset is generated using UWS; results may differ for other workflow systems. However, the features (checkpoint characteristics, corruption levels) are general concepts applicable across systems.

\paragraph{Construct Validity} Recovery success is defined as state completeness $\geq$ 50\% and no fatal errors. This threshold is pragmatic but arbitrary; different thresholds would yield different success rates.

\paragraph{Conclusion Validity} We use appropriate statistical methods: 5-fold cross-validation, 95\% confidence intervals, and non-parametric correlation (Spearman). Sample size (3,000 entries) provides adequate statistical power.

\subsection{Reproducibility}

All experiments are reproducible via our replication package:
\begin{itemize}
    \item Dataset: 3,000 entries in JSON and CSV formats
    \item Scripts: \texttt{predictive\_dataset\_generator.py}, \texttt{train\_predictive\_models.py}
    \item Environment: \texttt{requirements.txt} with pinned versions (scikit-learn 1.6.1, pandas 2.2.3)
    \item Docker: Containerized environment for consistent results
\end{itemize}

% ============================================================================
% RELATED WORK
% ============================================================================
\section{Related Work}
\label{sec:related}

UWS builds upon and differs from work in workflow orchestration, agent-based systems, context management, and developer productivity tools.

\subsection{Workflow Orchestration Systems}

Traditional workflow orchestration systems automate task execution:

\paragraph{Apache Airflow}~\cite{hazelwood2018airflow} provides DAG-based workflow definitions with database-backed state. While mature and scalable, Airflow targets data pipeline automation rather than interactive development workflows. Its state management requires PostgreSQL/MySQL infrastructure.

\paragraph{Temporal}~\cite{temporal2020} introduces event sourcing for workflow state, enabling deterministic replay and exactly-once execution. Temporal excels at distributed systems but requires significant infrastructure and does not integrate with git for state versioning.

\paragraph{Prefect}~\cite{prefect2021} emphasizes ``negative engineering''---making pipelines resilient without complex configuration. Like Airflow, Prefect focuses on data workflows rather than interactive AI-assisted development.

\paragraph{Dagster}~\cite{dagster2019} provides software-defined assets with strong typing and testing. Its asset-centric model differs from UWS's phase-based approach.

UWS differs by targeting \textit{interactive} AI-assisted workflows with git-native state management, eliminating external database requirements while enabling version control semantics.

\subsection{Agent-Based Systems}

Emerging LLM-based agent frameworks share UWS's multi-agent philosophy:

\paragraph{LangChain}~\cite{chase2022langchain} provides tools for building LLM applications with agents. LangGraph adds checkpointing for deterministic replay. However, LangChain focuses on LLM orchestration rather than developer workflow management, and checkpoints use external stores rather than git.

\paragraph{AutoGPT}~\cite{autogpt2023} demonstrates autonomous task completion but lacks robust context persistence. Restarting typically requires task re-specification.

\paragraph{CrewAI}~\cite{crewai2023} implements role-based multi-agent collaboration similar to UWS's agent design. However, CrewAI uses in-memory state with optional Redis persistence, not git-native storage.

\paragraph{AutoGen}~\cite{wu2023autogen} provides conversational agent frameworks with multi-turn interactions. Like LangChain, AutoGen focuses on LLM orchestration rather than developer workflow structure.

\paragraph{MetaGPT}~\cite{hong2024metagpt} applies software engineering principles to multi-agent collaboration. While aligned with UWS's philosophy, MetaGPT does not provide git-native state persistence or checkpoint-based recovery.

UWS uniquely combines multi-agent architecture with git-native state management, providing context persistence without external infrastructure.

\subsection{Context Management}

Research on context and memory in AI systems informs UWS's design:

\paragraph{Conversation Memory} Work on long-term memory for dialogue systems~\cite{xu2021beyond} addresses context retention across sessions. These approaches typically use vector databases or conversation summarization, differing from UWS's structured checkpoint approach.

\paragraph{Retrieval-Augmented Generation}~\cite{lewis2020retrieval} provides external knowledge access but does not address workflow state persistence.

\paragraph{Developer Interruption} Studies show developers require 15--25 minutes to recover focus after interruptions~\cite{mark2008cost, parnin2011programmer}. UWS directly addresses this recovery cost through structured context checkpointing.

\subsection{Developer Productivity Tools}

Tools enhancing developer productivity relate to UWS:

\paragraph{GitHub Copilot}~\cite{chen2021evaluating} provides inline code suggestions but does not maintain cross-session workflow state. Each session is independent.

\paragraph{IDE Session Management} Tools like VS Code's workspace history provide limited session recovery but lack structured workflow context.

\paragraph{DORA Metrics}~\cite{forsgren2018accelerate} and the SPACE framework~\cite{forsgren2021space} inform our understanding of developer productivity. UWS's recovery time metric aligns with DORA's ``time to restore'' dimension.

\subsection{Comparison Summary}

Table~\ref{tab:comparison} compares UWS with related systems:

\begin{table}[t]
    \centering
    \caption{System Comparison}
    \label{tab:comparison}
    \small
    \begin{tabular}{lccccc}
        \toprule
        & \rotatebox{90}{Git-Native} & \rotatebox{90}{Multi-Agent} & \rotatebox{90}{Checkpoints} & \rotatebox{90}{No Database} & \rotatebox{90}{Interactive} \\
        \midrule
        UWS & \checkmark & \checkmark & \checkmark & \checkmark & \checkmark \\
        Airflow & & & \checkmark & & \\
        Temporal & & & \checkmark & & \checkmark \\
        LangChain & & \checkmark & \checkmark & & \checkmark \\
        AutoGPT & & \checkmark & & & \checkmark \\
        CrewAI & & \checkmark & $\sim$ & & \checkmark \\
        \bottomrule
    \end{tabular}
\end{table}

UWS is unique in combining all five characteristics: git-native state management, multi-agent architecture, checkpoint-based recovery, no database requirements, and interactive workflow support.

\subsection{Positioning}

UWS occupies a distinct niche:
\begin{itemize}
    \item More structured than ad-hoc LLM interactions
    \item More interactive than batch workflow systems
    \item Simpler infrastructure than distributed orchestrators
    \item More persistent than existing agent frameworks
\end{itemize}

This positioning targets researchers and developers conducting long-term AI-assisted projects where context persistence is critical.

% ============================================================================
% CONCLUSION
% ============================================================================
\section{Conclusion}
\label{sec:conclusion}

We presented, to our knowledge, the first predictive models specifically for automated workflow context recovery in AI-assisted development, along with a synthetic benchmark enabling reproducible research on workflow resilience. Using the Universal Workflow System (UWS) as an experimental testbed, we demonstrated that recovery outcomes are predictable and identified key factors affecting success.

\subsection{Summary of Contributions}

Our work makes the following contributions:

\begin{enumerate}
    \item \textbf{Synthetic Benchmark}: A dataset of 3,000 annotated recovery scenarios with controlled corruption levels (0--90\%), enabling causal analysis impossible with observational data. Unlike existing datasets (KaVE, DevGPT), our benchmark provides ground-truth recovery outcomes under systematic state degradation.

    \item \textbf{Predictive Models}: Machine learning models achieving MAE of 1.1ms for recovery time prediction ($R^2=0.756$) and AUC-ROC of 0.912 for recovery success classification, demonstrating that workflow recovery is predictable.

    \item \textbf{Feature Importance}: Analysis revealing that corruption level dominates success/completeness prediction (63\% feature importance), while checkpoint-related features dominate time prediction (82\% combined importance).

    \item \textbf{Implementation}: Open-source testbed with comprehensive test suite (356 tests: 93\% core pass rate, 76\% experimental), benchmark scripts, and replication package.
\end{enumerate}

\subsection{Key Insights}

Our work reveals several insights for predictive modeling of workflow recovery:

\paragraph{Corruption Level is Dominant} For recovery success and state completeness, corruption level is the strongest predictor (63\% and 91\% feature importance respectively). This suggests that recovery system design should prioritize corruption resilience over other factors.

\paragraph{Time Prediction is Structural} Recovery time is primarily determined by checkpoint-related features (log size, count), not corruption level. This indicates that time complexity scales with state size, not failure severity.

\paragraph{Synthetic Benchmarks Enable Causal Analysis} Unlike observational datasets, our controlled methodology isolates variables, enabling definitive statements about which factors affect recovery. This approach is complementary to real-world observational studies.

\subsection{Limitations}

We acknowledge the following limitations:

\paragraph{Synthetic Data} Our dataset is programmatically generated with controlled parameters. While this enables causal analysis, high model performance on synthetic data does not guarantee real-world generalization. We explicitly scope our claims to this benchmark and identify validation with production data as critical future work.

\paragraph{Single-System Training} Models are trained exclusively on UWS-generated data. Features are specific to UWS's architecture (checkpoint counts, handoff sizes). Transfer learning to other workflow systems requires further investigation.

\paragraph{Construct Validity} We measure \emph{system recovery} (script execution, file parsing), not \emph{developer recovery} (cognitive re-engagement). The relationship between technical recovery time and developer productivity requires a user study to establish.

\paragraph{Corruption Model Simplicity} Our corruption simulation uses byte-level random corruption. Real-world failures may exhibit different patterns (e.g., incomplete writes, sector-level failures). More sophisticated fault injection models are future work.

\paragraph{No User Study} We do not measure developer productivity directly. Claims about practical benefit are based on the assumption that faster technical recovery enables faster cognitive re-engagement, which requires empirical validation.

\subsection{Future Work}

Several directions extend this work:

\paragraph{Real-World Validation} Deploying models in production environments and collecting actual failure data would validate external generalizability. Online learning approaches could adapt models to real-world failure patterns.

\paragraph{Cross-System Generalization} Testing the same predictive methodology on other workflow systems (LangGraph, AutoGen, CrewAI) would strengthen external validity. Our framework comparison benchmark provides a starting point for this investigation.

\paragraph{SHAP Analysis} Applying SHapley Additive exPlanations (SHAP) would provide deeper interpretability into model predictions, enabling understanding of individual predictions rather than aggregate feature importance.

\paragraph{User Studies} Controlled experiments measuring developer productivity with and without recovery prediction would establish the practical value of predictive models.

\paragraph{Adaptive Checkpointing} Using recovery prediction models to optimize checkpoint frequency---creating more checkpoints when predicted recovery success is low---could improve system resilience proactively.

\subsection{Availability}

UWS is open source and available for adoption:
\begin{itemize}
    \item Source code: GitHub (anonymized for review)
    \item Replication package: Zenodo (DOI provided after acceptance)
    \item Documentation: Comprehensive guides included
\end{itemize}

We welcome community contributions and feedback to advance context-resilient AI-assisted development.

\section*{Data Availability}
A complete replication package is available containing: (1) full source code for UWS, (2) all test suites (unit, integration, E2E, performance), (3) benchmark scripts and raw result data, (4) analysis code for statistical computations, and (5) instructions for reproducing all experiments. The package is available at an anonymized repository for review and will be archived on Zenodo with a DOI upon acceptance. All experiments can be reproduced with a single command: \texttt{./tests/benchmarks/benchmark\_runner.sh}.

\subsection{Closing Remarks}

As AI-assisted development workflows become more complex and long-running, predicting and optimizing recovery from failures becomes increasingly important. Our work demonstrates that workflow recovery is predictable, with machine learning models achieving AUC=0.912 for success classification on our synthetic benchmark. By releasing our dataset and methodology, we enable the research community to build upon this foundation, developing more sophisticated models and validating generalizability across diverse workflow systems.

The key insight from our feature importance analysis---that corruption level dominates success prediction while checkpoint characteristics dominate time prediction---provides actionable guidance for workflow system designers: prioritize corruption resilience mechanisms and manage checkpoint complexity for optimal recovery.


% ============================================================================
% ACKNOWLEDGMENTS (hidden for review)
% ============================================================================
% \begin{acks}
% Acknowledgments here after acceptance.
% \end{acks}

% ============================================================================
% REFERENCES
% ============================================================================
\bibliographystyle{ACM-Reference-Format}
\bibliography{references}

% Balance columns on last page
\balance

\end{document}
