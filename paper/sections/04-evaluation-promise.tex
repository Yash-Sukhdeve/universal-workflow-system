% ============================================================================
% EVALUATION - PROMISE 2026 VERSION
% Focused on Predictive Model Evaluation
% ============================================================================
\section{Evaluation}
\label{sec:evaluation}

We evaluate our predictive models for workflow context recovery through comprehensive experiments addressing three research questions.

\subsection{Research Questions}

\begin{itemize}
    \item \textbf{RQ1 (Recovery Time Prediction)}: How accurately can we predict context recovery time from workflow characteristics?
    \item \textbf{RQ2 (Recovery Success Prediction)}: How accurately can we predict whether recovery will succeed under given conditions?
    \item \textbf{RQ3 (Feature Importance)}: Which workflow characteristics most strongly influence recovery outcomes?
\end{itemize}

\subsection{Dataset Construction}

\paragraph{Scenario Generation} We generated 1,000 unique recovery scenarios by systematically varying workflow parameters:

\begin{itemize}
    \item \textbf{Checkpoint count}: 1, 5, 10, 25, 50, 100, 200
    \item \textbf{State complexity}: minimal, low, medium, high, complex
    \item \textbf{Project type}: ML pipeline, web development, research, DevOps, data engineering, LLM application, mixed
    \item \textbf{Corruption level}: 0\%, 5\%, 10\%, 25\%, 50\%, 75\%, 90\%
    \item \textbf{Interruption type}: clean, abrupt, crash, timeout
    \item \textbf{Handoff size}: small (500 chars), medium (2,000), large (8,000), very large (25,000)
\end{itemize}

\paragraph{Ground Truth Collection} For each scenario, we executed three independent trials, measuring:
\begin{itemize}
    \item Recovery time (milliseconds)
    \item Recovery success (binary)
    \item State completeness (percentage)
\end{itemize}

The final dataset contains 3,000 entries (1,000 scenarios $\times$ 3 trials) with 18 features per entry.

\paragraph{Dataset Statistics} Table~\ref{tab:dataset-stats} summarizes the dataset characteristics:

\begin{table}[h]
    \centering
    \caption{Predictive Dataset Statistics}
    \label{tab:dataset-stats}
    \begin{tabular}{lr}
        \toprule
        \textbf{Metric} & \textbf{Value} \\
        \midrule
        Total entries & 3,000 \\
        Unique scenarios & 1,000 \\
        Features & 18 \\
        Recovery time (mean) & 29.4ms \\
        Recovery time (std dev) & 3.5ms \\
        Success rate & 85.3\% \\
        \bottomrule
    \end{tabular}
\end{table}

\subsection{Experimental Setup}

\paragraph{Models} We evaluated four model families using scikit-learn~\cite{pedregosa2011scikit}:
\begin{itemize}
    \item \textbf{Linear models}: Linear Regression, Ridge Regression, Logistic Regression
    \item \textbf{Ensemble models}: Random Forest~\cite{breiman2001random}, Gradient Boosting~\cite{friedman2001greedy}
\end{itemize}

\paragraph{Validation} We used 5-fold cross-validation on 80\% training data, with 20\% held out for final testing. All experiments used random seed 42 for reproducibility.

\paragraph{Metrics} For regression: MAE, RMSE, $R^2$. For classification: Accuracy, F1-score, AUC-ROC. All metrics include 95\% confidence intervals.

\subsection{RQ1: Recovery Time Prediction}

Table~\ref{tab:regression-results} presents recovery time prediction results:

% Auto-generated table for PROMISE 2026
\begin{table}[t]
    \centering
    \caption{Recovery Time Prediction Results (5-fold CV)}
    \label{tab:regression-results}
    \begin{tabular}{lrrr}
        \toprule
        \textbf{Model} & \textbf{MAE (ms)} & \textbf{RMSE (ms)} & \textbf{$R^2$} \\
        \midrule
        Linear Regression & 2.2 [2.1, 2.3] & 3.1 & 0.152 \\
        Ridge Regression & 2.2 [2.1, 2.3] & 3.1 & 0.152 \\
        Random Forest & 1.2 [1.1, 1.3] & 1.9 & 0.718 \\
        Gradient Boosting & 1.1 [1.0, 1.2] & 1.7 & 0.756 \\
        \bottomrule
    \end{tabular}
\end{table}


\paragraph{Findings} Gradient Boosting achieves the best performance with MAE of 1.1ms (95\% CI: [1.0, 1.2]) and $R^2 = 0.756$. This means:
\begin{itemize}
    \item Predictions are within 1.1ms of actual recovery time on average
    \item The model explains 75.6\% of variance in recovery time
    \item Linear models perform poorly ($R^2 = 0.152$), suggesting non-linear relationships
\end{itemize}

\paragraph{Practical Significance} With mean recovery time of 29.4ms, MAE of 1.1ms represents 3.7\% error---sufficient for practical applications like adaptive checkpointing and resource planning.

\subsection{RQ2: Recovery Success Prediction}

Table~\ref{tab:classification-results} presents recovery success prediction results:

% Auto-generated table for PROMISE 2026
\begin{table}[t]
    \centering
    \caption{Recovery Success Prediction Results (5-fold CV)}
    \label{tab:classification-results}
    \begin{tabular}{lrrr}
        \toprule
        \textbf{Model} & \textbf{Accuracy} & \textbf{F1 Score} & \textbf{AUC-ROC} \\
        \midrule
        Logistic Regression & 0.856 & 0.917 & 0.904 \\
        Random Forest & 0.845 & 0.909 & 0.907 \\
        Gradient Boosting & 0.851 & 0.911 & 0.912 \\
        \bottomrule
    \end{tabular}
\end{table}


\paragraph{Findings} All models achieve strong predictive performance:
\begin{itemize}
    \item Gradient Boosting: AUC-ROC = 0.912, F1 = 0.911
    \item Logistic Regression: AUC-ROC = 0.904, F1 = 0.917
    \item Random Forest: AUC-ROC = 0.907, F1 = 0.909
\end{itemize}

\paragraph{Confusion Matrix Analysis} On the test set (600 samples), Gradient Boosting achieved:
\begin{itemize}
    \item True Negatives: 51 (correctly predicted failures)
    \item False Positives: 37 (predicted success, actual failure)
    \item False Negatives: 49 (predicted failure, actual success)
    \item True Positives: 463 (correctly predicted successes)
\end{itemize}

The 5.8\% false positive rate indicates the model may occasionally overestimate recovery success under adverse conditions.

\subsection{RQ3: Feature Importance Analysis}

We analyzed feature importance using two methods: (1) Spearman correlation with target variables, and (2) model-based feature importance from tree ensembles.

\paragraph{Recovery Time Features} The top predictors of recovery time are:
\begin{enumerate}
    \item \textbf{Handoff document size} ($r = 0.531$, $p < 0.001$): Larger handoff documents increase recovery time
    \item \textbf{Checkpoint count} ($r = 0.318$, $p < 0.001$): More checkpoints require more parsing
    \item \textbf{Checkpoint log size} ($r = 0.318$, $p < 0.001$): Directly correlated with checkpoint count
\end{enumerate}

\paragraph{Recovery Success Features} The top predictors of recovery success are:
\begin{enumerate}
    \item \textbf{Corruption level} ($r = -0.475$, $p < 0.001$): Higher corruption strongly reduces success probability
    \item \textbf{Interruption type}: Crash and timeout interruptions reduce success rates
    \item \textbf{Phase progress}: Earlier phases show slightly higher recovery success
\end{enumerate}

\paragraph{Model-Based Importance} Random Forest feature importance confirms these findings:
\begin{itemize}
    \item For recovery time: checkpoint\_count (23\%), handoff\_chars (18\%), checkpoint\_log\_size (17\%)
    \item For recovery success: corruption\_level (31\%), interruption\_type (22\%), phase\_progress (15\%)
\end{itemize}

\subsection{State Completeness Prediction}

We also evaluated models for predicting state completeness percentage:

\begin{table}[h]
    \centering
    \caption{State Completeness Prediction Results}
    \label{tab:completeness-results}
    \begin{tabular}{lrrr}
        \toprule
        \textbf{Model} & \textbf{MAE (\%)} & \textbf{RMSE (\%)} & \textbf{$R^2$} \\
        \midrule
        Random Forest & 8.51 & 13.53 & 0.680 \\
        Gradient Boosting & 8.79 & 11.45 & 0.770 \\
        \bottomrule
    \end{tabular}
\end{table}

Gradient Boosting achieves MAE of 8.79\% for state completeness prediction ($R^2 = 0.770$), enabling estimation of how much context will be recoverable under given conditions.

\subsection{Threats to Validity}

\paragraph{Internal Validity} Scenario generation uses controlled parameter variation, which may not capture all real-world correlations. We mitigate this through systematic coverage of the parameter space combined with random sampling.

\paragraph{External Validity} Our dataset is generated using UWS; results may differ for other workflow systems. However, the features (checkpoint characteristics, corruption levels) are general concepts applicable across systems.

\paragraph{Construct Validity} Recovery success is defined as state completeness $\geq$ 50\% and no fatal errors. This threshold is pragmatic but arbitrary; different thresholds would yield different success rates.

\paragraph{Conclusion Validity} We use appropriate statistical methods: 5-fold cross-validation, 95\% confidence intervals, and non-parametric correlation (Spearman). Sample size (3,000 entries) provides adequate statistical power.

\subsection{Reproducibility}

All experiments are reproducible via our replication package:
\begin{itemize}
    \item Dataset: 3,000 entries in JSON and CSV formats
    \item Scripts: \texttt{predictive\_dataset\_generator.py}, \texttt{train\_predictive\_models.py}
    \item Environment: \texttt{requirements.txt} with pinned versions (scikit-learn 1.6.1, pandas 2.2.3)
    \item Docker: Containerized environment for consistent results
\end{itemize}
